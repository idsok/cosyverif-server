\PassOptionsToPackage{svgnames}{xcolor}
\documentclass[t, english]{beamer}
\usepackage[utf8]{inputenc}
\usepackage[T1]{fontenc}
\usepackage{tikz}
\usepackage[default]{comfortaa}
\usepackage{pgfpages}
\usepackage{listings}
\usepackage{appendixnumberbeamer}

\usetikzlibrary{trees, spy, decorations.text}

\lstset{
  basicstyle=\ttfamily\scriptsize\color{black},
  showstringspaces=false,
  commentstyle=\color{red},
  keywordstyle=\color{blue},
  language=bash,
  backgroundcolor=\color{white},
}

%\pgfpagesuselayout{8 on 1}[a4paper, border shrink=5mm]
%\setbeameroption{show notes on second screen=bottom}
\setbeameroption{hide notes}
%%\setbeameroption{show notes}

\setbeamertemplate{note page}[plain]
\beamertemplatenavigationsymbolsempty

\synctex=1
\hypersetup{pdfpagemode=UseNone} % don't show bookmarks on initial view
\usetheme{default}
\usefonttheme{professionalfonts}

\setbeamertemplate{footline}{%
\raisebox{5pt}{%
\makebox[\paperwidth]{%
\hfill\makebox[25pt]{%
\scriptsize\insertframenumber/\inserttotalframenumber%
}%
}%
}\hspace*{5pt}%
}

\begin{document}

\title{Etude et réalisation du serveur de la nouvelle plate-forme CosyVerif}
\author{Idrissa SOKHONA}
\date{29 Septembre 2014}

\begin{frame}
\begin{center}
\Huge Conception et réalisation du serveur de la nouvelle plate-forme CosyVerif

\par\vfill
\normalsize
\textsf{%
Idrissa SOKHONA \\
\url{idrissa.sokhona@etu.upmc.fr} \\
Université Pierre et Marie Curie \\
Master d'Informatique \\
Spécialité SAR, Parcours SRETR
}
\par\vfill
\begin{tikzpicture}
    \node
      [anchor=north]
      at (1,3)
      {\includegraphics[height=1cm]{img/lipn}};
    \node
      [anchor=north]
      at (6.5,3)
      {\includegraphics[height=1cm]{img/lsv}};
  \end{tikzpicture}
\end{center}
\end{frame}

\begin{frame}[c]
  \frametitle{Contexte}
%  \begin{minipage}{1\textwidth}
%  	\only <1>{%
%                \begin{tikzpicture}
%                      \node
%                          [anchor=north]
%                          at (0,0)
%                          {\includegraphics[width=10.5cm, height=4cm]{img/1-chercheur}};
%                 \end{tikzpicture}
%            }%
%            
%            \only <2-7>{%
%                \begin{tikzpicture}
%                      \node
%                          [anchor=north]
%                          at (0,0)
%                          {\includegraphics[width=10.5cm, height=4cm]{img/2-chercheur}};
%                 \end{tikzpicture}
%            }%
%            
%            \only <8>{%
%                \begin{tikzpicture}
%                      \node
%                          [anchor=north]
%                          at (0,0)
%                          {\includegraphics[width=10.5cm, height=4cm]{img/3-chercheur}};
%                 \end{tikzpicture}
%            }%
%            
%            \only <9>{%
%                \begin{tikzpicture}
%                      \node
%                          [anchor=north]
%                          at (0,0)
%                          {\includegraphics[width=10.5cm, height=4cm]{img/4-chercheur}};
%                 \end{tikzpicture}
%            }%
% 
%    \end{minipage}
%   
%   \hrule
%  
%  \begin{minipage}{1\textwidth}
%    
%         \begin{itemize}
%     	    \item <3->Installer
%     	    \item <4->Prise en main
%	    \item <5->Format d'entrée
%     	    \item <6->Format de sortie
%     	    \item <7->Analyse de résultats
%         \end{itemize}
%   \end{minipage}
%
  \centering
  \begin{tikzpicture}[
    spy using outlines={%
      circle,
      magnification=5,
      size=5cm,
      connect spies
    }
  ]
  \useasboundingbox [] (-5,3) rectangle (5,-3);
    \only<8>{
    \node (cosy)
       at (0, 0)
       {\includegraphics[width=6cm]{img/cosyverif}};
    }
    \node (tool-1)
       at (-1, .4)
       {\includegraphics[height=1cm]{img/tool-0}};
    \node (tool-2)
       at ( 0, .4)
       {\includegraphics[height=1cm]{img/tool-3}};
    \node (tool-3)
       at ( 1, .4)
       {\includegraphics[height=1cm]{img/tool-4}};
    \only<2->{
    \begin{scope}
      \clip [rounded corners=.5cm]
        (-5,3) -- (-1,3) .. controls (-3,1) .. (-5,0) -- cycle;
      \node [anchor=north west]
        at (-5, 3)
        {\includegraphics[width=4cm]{img/research}};
    \end{scope}
    }
    \only<3->{
    \begin{scope}
      \clip [rounded corners=.5cm]
        (5,3) -- (1,3) .. controls ( 3,1) .. ( 5,0) -- cycle;
      \node [anchor=north east]
        at ( 5, 3)
        {\includegraphics[width=4cm]{img/teaching}};
    \end{scope}
    }
    \only<4->{
    \begin{scope}
      \clip [rounded corners=.5cm]
        (-5,-3) -- (-1,-3) .. controls (-3,-1) .. (-5,0) -- cycle;
      \node [anchor=south west]
        (domain-industry)
        at (-5, -3)
        {\includegraphics[width=4cm]{img/industry}};
    \end{scope}
    }
    \only<5->{
    \begin{scope}
      \clip [rounded corners=.5cm]
        (5,-3) -- (1,-3) .. controls ( 3,-1) .. ( 5,0) -- cycle;
      \node [anchor=south east]
        at ( 5, -3)
        {\includegraphics[width=4cm]{img/communication}};
    \end{scope}
    }
    % Ajouter les domaines engineering et communication
    %
    %
    \only<6>{
    \begin{scope}
      \clip [rounded corners=.5cm]
        (-2.5,1.5) rectangle (-3.5,.5);
      \node (developer)
        at (-3,1)
        {\includegraphics[height=1cm]{img/andre.jpg}};
    \end{scope}
      \draw [draw=red, ->, line width=2pt] (developer) -- (tool-1);
    }
    \only<7>{
      \draw [draw=red, ->, line width=2pt] (tool-2) -- (-2.5,-1);
      \draw [decorate,
        decoration={text along path, text = {|\small|installation},
                    text format delimiters={|}{|},
                    text align=center}]
        ([shift=(0:0cm)]-2,0) arc (180:90:2cm);
      \draw [decorate,
        decoration={text along path, text = {|\small|interaction},
                    text format delimiters={|}{|},
                    text align=center}]
        ([shift=(0:0cm)]0,2) arc (90:0:2cm);
      \draw [decorate,
        decoration={text along path, text = {|\small|formats},
                    text format delimiters={|}{|},
                    text align=center}]
        ([shift=(0:0cm)]-2,0) arc (180:270:2cm);
      \draw [decorate,
        decoration={text along path, text = {|\small|analyse},
                    text format delimiters={|}{|},
                    text align=center}]
        ([shift=(0:0cm)]0,-2) arc (270:360:2cm);
    }
    % Ajouter d'autres utilisateurs / dévelopeurs
    %
%    \draw [<->] (domain-research) -- (tool)
%     node [near start, above] {CAMI}
%     node [near end  , above] {CAMI};
    %
%    \only<2>{\spy[red!70!black] on (-4,1.5) in node [right] at (-2.5,0);}
%    \only<3>{\spy[red!70!black] on ( 4,1.5) in node [left ] at ( 2.5,0);}
%    \only<4>{\spy[red!70!black] on ( 3,1.5) in node [left ] at ( 2.5,0);}
  \end{tikzpicture}
   \note{
    \begin{itemize}
     \item CosyVerif est une collection d'outils
     \item Ces outils sont utilisés à plusieurs occasions :
       \begin{itemize}
         \item Recherche, par exemple pour comparer ses résultats
         \item Enseignement, par exemple le cours ??? de Fabrice Kordon
               en M2 SAR
         \item Projets industriels, par exemple lors du projet européen MIDAS
         \item Communication, par exemple lors des démonstrations d'outils
               dans les conférences, ou les portes ouvertes du LSV
       \end{itemize}
     \item Ces outils sont écrits par des chercheurs
     \item Ces outils sont dédiés à des calculs très pointus
     \item Mais les chercheurs n'ont pas de temps à consacrer à leur peaufinage
     \item Les outils sont donc généralement difficiles à utiliser :
       \begin{itemize}
         \item Difficulté d'installation, par exemple GreatSPN a nécessité
           deux demies-journées pour arriver à l'installer
         \item Difficulté de prise en main, souvent en ligne de commande,
           aide plus ou moins existante
         \item Formats d'entrée très variables, souvent propre à l'outil,
           par exemple Prod, GreatSPN... un format standardisé PNML pas
           toujours supporté
         \item Format de sortie propre à chaque outil, souvent texte
         \item Analyse des résultats pas évidente, par exemple vérifier la
           présence de deadlock avec Prod ne répond pas juste oui/non
       \end{itemize}
     \item CosyVerif aide à utiliser les outils :
       \begin{itemize}
         \item En empaquetant les outils
         \item En fournissant un format unique d'entrée / sortie
         \item En fournissant une méthode unique d'interaction avec les
           outils
         \item En fournissant différents clients permettant d'interagir avec
           le serveur
       \end{itemize}
   \end{itemize}
   Transition: comment?
   }
 \end{frame}

\begin{frame}
\only<1-2>{
  \frametitle{État actuel}
}\only<3-7>{
  \frametitle{Problèmes}
}\only<8->{
  \frametitle{Évolutions}
}
  \centering
\begin{tikzpicture}
  \useasboundingbox [] (.5,1) rectangle (10,8.5);
\node (c1) at (2,3)
  {\includegraphics[width=3cm,height=3cm]{img/screen}};
\node (c2) at (2,7)
  {\includegraphics[width=3cm,height=3cm]{img/screen}};
\node at (2,5)
  {Clients};
\node (s) at (7,5)
  {\includegraphics[width=3cm,height=3cm]{img/screen}};
\only<5-7>{
\node [anchor=center]
  at (s.south)
  {\includegraphics[width=1cm]{img/java2}};
}
\only<8->{
\node [anchor=center]
  at (s.south)
  {\includegraphics[width=1cm]{img/php.png}};
}
\only<7->{
  \node [anchor=north west] at (s.east)
  {\includegraphics[width=1cm]{img/depot2}?};
}
\only<6->{
  \node [anchor=north west] at (s.north east)
  {\includegraphics[width=1cm]{img/users}?};
}
\node at (7,3)
  {Server};
\only<1-4>{
\node
  at (2, 3.3)
  {\includegraphics[width=2.8cm,height=2.2cm]{img/coloane-screenshot}};
}
\only<4-7>{
\node (coloane)
  at (2, 3.3)
  {\includegraphics[width=2.8cm,height=2.2cm]{img/coloane-deprecated}};
\node [anchor=north]
  at (coloane.south)
  {\includegraphics[width=1cm]{img/eclipse}};
}
\only<7->{
\node (coloane)
  at (2, 3.3)
  {\color{white}?};
}
\node
  at (2, 7.3)
  {\includegraphics[width=2.8cm,height=2.2cm]{img/clitor}};
\node (t1)
  at (6.5, 5)
  {\includegraphics[height=1cm]{img/tools-1}};
\node (t2)
  at (7.5, 5.5)
  {\includegraphics[height=1cm]{img/tools-2}};
\only<1>{
\draw [<->, line width=2pt]
  (c1) -- (s);
\draw [<->, line width=2pt]
  (c2) -- (s);
}
\only<3-7>{
\draw [<->, line width=2pt]
  (c1) -- (s)
  node [midway, below, sloped] {SOAP};
\draw [<->, line width=2pt]
  (c2) -- (s)
  node [midway, above, sloped] {SOAP};
}
\only<8->{
\draw [<->, line width=2pt]
  (c1) -- (s)
  node [midway, below, sloped] {REST};
\draw [<->, line width=2pt]
  (c2) -- (s)
  node [midway, above, sloped] {REST};
}
\only<2>{
\draw [<->, line width=2pt]
  (c1) -- (s.west)
  node [midway, below, sloped] {GrML};
\draw [<->, line width=2pt]
  (c2) -- (s.west)
  node [midway, above, sloped] {GrML};
\draw [<->, color=DarkGreen, line width=2pt]
  (s.west) -- (t2.center)
  node [midway, above, sloped] {
    \tiny\color{white} GrML $\leftrightarrow$ ?
  };
}
\end{tikzpicture}

\note{
  \begin{itemize}
    \item Architecture client/serveur
    \item Deux types de clients : ligne de commande et client/éditeur
      graphique
    \item Un serveur qui embarque les outils
    \item Format d'échange GrML
    \item Problème \#1: protocole de communication SOAP
      \begin{itemize}
        \item Adapté aux projets industriels
        \item Lourd, nécessite une bibliothèque complexe
        \item Clients difficiles à écrire, par exemple le client en ligne de
          commande écrit en PHP a rencontré des difficultés
      \end{itemize}
    \item Problème \#2: maintenance du client Coloane
      \begin{itemize}
        \item Plug-in Eclipse
        \item Pas de permanent disposant de ces compétences dans CosyVerif
        \item Coût d'entrée trop élevé pour les stagiaires
      \end{itemize}
    \item Problème \#3: évolution du serveur
      \begin{itemize}
        \item Écrit en Java
        \item Repose sur des bibliothèques complexes (WebServices)
        \item Pas de permanent disposant de ces compétences dans CosyVerif
        \item Des fonctionnalités simples et nécessaires (utilisateurs,
          dépôt) prennent trop de temps à développer
      \end{itemize}
    \item Évolution \#1: passer à REST
      \begin{itemize}
        \item
      \end{itemize}
    \item Évolution \#2: passer à un client et éditeur Web
      \begin{itemize}
        \item Ce n'est pas le sujet de mon stage
        \item La partie éditeur est en cours de réalisation par un autre
          stagiaire
      \end{itemize}
    \item Évolution \#3: passer le serveur en PHP
      \begin{itemize}
        \item But: maintenance, car plusieurs membres permanents de
          CosyVerif maitrisent le PHP
        \item But: ajout plus simple de fonctionnalisés
      \end{itemize}
    \item Évolution \#4: ajouter la notion d'utilisateurs
    \item Évolution \#5: ajouter la gestion d'un dépôt de formalismes et
      modèles, de manière similaire à PNMLWeb ou PetriWeb
    \item Contrainte \#1: écrire un code maintenable
    \item Contrainte \#2: avoir une très bonne couverture de code par les
      tests
  \end{itemize}
}
\end{frame}


\begin{frame}[t,fragile]
\only<1>{
  \frametitle{Réalisation}
}\only<2-4>{
  \frametitle{Création d'un utilisateur}
}\only<5-6>{
  \frametitle{Ajout de modèle}
}\only<7>{
  \frametitle{Lecture de modèle}
}\only<8>{
  \frametitle{Modification de modèle}
}\only<9-11>{
  \frametitle{Suppression de modèle}
}\only<12>{
  \frametitle{Lecture du collection de modèles}
}\only<13-15>{
  \frametitle{Création d'un autre utilisateur}
}\only<16-22>{
  \frametitle{Passage à un compte privé}
}\only<23-24>{
  \frametitle{Création de projet}
}\only<25-26>{
  \frametitle{Déplacement de modèle}
}\only<27-29>{
  \frametitle{Invitation d'utilisateur}
}\only<30->{
  \frametitle{Autres fonctionnalités implémentées}
}


\begin{columns}[T] % align columns
\begin{column}{.65\textwidth}
  \lstset{
    breaklines=true,
    postbreak=\raisebox{0ex}[0ex][0ex]{\ensuremath{\color{red}\hookrightarrow\space}}
  }
\begin{semiverbatim}
\scriptsize
\only<1>{
  Requêtes/Réponses
}\only<3>{
  POST http://rest.cosyverif.org/users/isokhona
            (request_type=create)
  201 CREATED
}\only<4>{
  POST http://rest.cosyverif.org/users/isokhona
  (request_type=activate 
   key=eda7e889177963b0d)
  200 OK
}\only<6>{
  POST .../users/isokhona/models/philosophers
              (request_type=create)
  201 CREATED
}\only<7>{
  GET .../users/isokhona/models/philosophers
  200 OK & data
}\only<8>{
  PUT .../users/isokhona/models/philosophers
  200 OK
}\only<10>{
  POST .../users/isokhona/models/deletetest
            (request_type=create)
  201 CREATED
}\only<11>{
  DELETE .../users/isokhona/models/deletetest
  204 NO CONTENT
}\only<12>{
  GET .../users/isokhona/models
  200 OK & data
}\only<14>{
  POST http://rest.cosyverif.org/users/alban
            (request_type=create)
  201 CREATED
}\only<15>{
  POST http://rest.cosyverif.org/users/alban
  (request_type=activate
   key=tztz74ddiii45)
  200 OK
}
\only<16>{
 isokhona est publique
}\only<17>{
  isokhona
  
  GET .../users/isokhona/models/philosophers
  200 OK & data
}\only<18>{
  alban
  
  GET .../users/isokhona/models/philosophers
  200 OK & data
}\only<19>{
  PUT http://rest.cosyverif.org/users/isokhona
  (is_public=false)
  200 OK
}\only<20>{
 isokhona est privé
}\only<21>{
  isokhona
  
  GET .../users/isokhona/models/philosophers
  200 OK & data
}\only<22>{
  alban
  
  GET .../users/isokhona/models/philosophers
  403 FORBIDDEN
}\only<24>{
  POST .../users/isokhona/projects/vfsr_groupe1
       (request_type = create)
  201 CREATED
}\only<26>{
  POST .../projects/vfsr_groupe1/models/philosophers
    (request_type = move, 
     source = /users/isokhona/models/philosophers)
  201 CREATED
}\only<28>{
  isokhona
  
  POST .../projects/vfsr_groupe1/invitations/alban
  201 CREATED
}\only<29>{
  alban
  
  PUT .../users/alban/invitations/vfsr_groupe1
  	(status = ACK)
  200 OK
}\only<30->{
  - Changement de permissions
}
\only<31->{
  - Recherche de ressources
}

\end{semiverbatim}


\end{column}%
\hfill%
\begin{column}{.34\textwidth}
% http://www.texample.net/tikz/examples/filesystem-tree/

\tikzstyle{every node}=[draw=black,thick,anchor=west,minimum width=2cm,minimum height=2em]
\tikzstyle{selected}=[draw=red,fill=red!30]
\tikzstyle{optional}=[dashed,fill=gray!50]
\scriptsize
\begin{tikzpicture}[
every node={draw=black,thick,anchor=west},
selected={draw=red,fill=red!30},
grow via three points={one child at (-.5,-0.7) and
two children at (-.5,-0.7) and (-.5,-1.4)},
edge from parent path={(\tikzparentnode.south west) |- (\tikzchildnode.west)}]
\only<1-2>{%
	\node {/}
		child { node {users}}
		child { node {projects}};
}%

\only<3-4>{%
	\node {/}
		child { node {users}
			child { node [selected] {isokhona}
				child { node {formalisms}}
				child { node {models}}
				child { node {services}}
				child { node {executions}}
			}
		}
		child [missing] {}
		child [missing] {}
		child [missing] {}
		child [missing] {}
		child [missing] {}
		child { node {projects}};
}%

\only<5>{%
	\node {/}
		child { node {users}
			child { node [selected] {isokhona}
				child { node {formalisms}}
				child { node [selected] {models}}
				child { node {services}}
				child { node {executions}}
			}
		}
		child [missing] {}
		child [missing] {}
		child [missing] {}
		child [missing] {}
		child [missing] {}
		child { node {projects}};
}%

\only<6-8>{%
	\node {/}
		child { node {users}
			child { node {isokhona}
				child { node {formalisms}}
				child { node {models}
				   child { node [selected] {philosophers}}
				}
				child [missing] {}
				child { node {services}}
				child { node {executions}}
			}
		}
		child [missing] {}
		child [missing] {}
		child [missing] {}
		child [missing] {}
		child [missing] {}
		child [missing] {}
		child { node {projects}};
}%

\only<9>{%
	\node {/}
		child { node {users}
			child { node {isokhona}
				child { node {formalisms}}
				child { node {models}
				   child { node {philosophers}}
				}
				child [missing] {}
				child { node {services}}
				child { node {executions}}
			}
		}
		child [missing] {}
		child [missing] {}
		child [missing] {}
		child [missing] {}
		child [missing] {}
		child [missing] {}
		child { node {projects}};
}%

\only<10>{%
	\node {/}
		child { node {users}
			child { node {isokhona}
				child { node {formalisms}}
				child { node {models}
				   child { node {philosophers}}
				   child { node [selected] {deletetest}}
				}
				child [missing] {}
				child [missing] {}
				child { node {services}}
				child { node {executions}}
			}
		}
		child [missing] {}
		child [missing] {}
		child [missing] {}
		child [missing] {}
		child [missing] {}
		child [missing] {}
		child [missing] {}
		child { node {projects}};
}%

\only<11>{%
	\node {/}
		child { node {users}
			child { node {isokhona}
				child { node {formalisms}}
				child { node {models}
				   child { node {philosophers}}
				}
				child [missing] {}
				child { node {services}}
				child { node {executions}}
			}
		}
		child [missing] {}
		child [missing] {}
		child [missing] {}
		child [missing] {}
		child [missing] {}
		child [missing] {}
		child { node {projects}};
}%

\only<12>{%
	\node {/}
		child { node {users}
			child { node {isokhona}
				child { node {formalisms}}
				child { node [selected]{models}
				   child { node {philosophers}}
				}
				child [missing] {}
				child { node {services}}
				child { node {executions}}
			}
		}
		child [missing] {}
		child [missing] {}
		child [missing] {}
		child [missing] {}
		child [missing] {}
		child [missing] {}
		child { node {projects}};
}%

\only<13>{%
	\node {/}
		child { node {users}
			child { node {isokhona}
				child { node {formalisms}}
				child { node {models}
				   child { node {philosophers}}
				}
				child [missing] {}
				child { node {services}}
				child { node {executions}}
			}
		}
		child [missing] {}
		child [missing] {}
		child [missing] {}
		child [missing] {}
		child [missing] {}
		child [missing] {}
		child { node {projects}};
}%

\only<14-15>{%
	\node {/}
		child { node {users}
			child { node {isokhona}}
			child { node [selected] {alban}
				child { node {formalisms}}
				child { node {models}}
				child { node {services}}
				child { node {executions}}
			}
		}
		child [missing] {}
		child [missing] {}
		child [missing] {}
		child [missing] {}
		child [missing] {}
		child [missing] {}
		child { node {projects}};
}%

\only<16>{%
	\node {/}
		child { node {users}
			child { node [selected]{isokhona}
				child { node {formalisms}}
				child { node  {models}
				   child { node {philosophers}}
				}
				child [missing] {}
				child { node {services}}
				child { node {executions}}
			}
			child [missing] {}
			child [missing] {}
			child [missing] {}
			child [missing] {}
			child [missing] {}
			child { node {alban}}
		}
		child [missing] {}
		child [missing] {}
		child [missing] {}
		child [missing] {}
		child [missing] {}
		child [missing] {}
		child [missing] {}
		child { node {projects}};
}%

\only<17-18>{%
	\node {/}
		child { node {users}
			child { node {isokhona}
				child { node {formalisms}}
				child { node  {models}
				   child { node [selected]{philosophers}}
				}
				child [missing] {}
				child { node {services}}
				child { node {executions}}
			}
			child [missing] {}
			child [missing] {}
			child [missing] {}
			child [missing] {}
			child [missing] {}
			child { node {alban}}
		}
		child [missing] {}
		child [missing] {}
		child [missing] {}
		child [missing] {}
		child [missing] {}
		child [missing] {}
		child [missing] {}
		child { node {projects}};
}%

\only<19-20>{%
	\node {/}
		child { node {users}
			child { node [selected]{isokhona}
				child { node {formalisms}}
				child { node  {models}
				   child { node {philosophers}}
				}
				child [missing] {}
				child { node {services}}
				child { node {executions}}
			}
			child [missing] {}
			child [missing] {}
			child [missing] {}
			child [missing] {}
			child [missing] {}
			child { node {alban}}
		}
		child [missing] {}
		child [missing] {}
		child [missing] {}
		child [missing] {}
		child [missing] {}
		child [missing] {}
		child [missing] {}
		child { node {projects}};
}%

\only<21-22>{%
	\node {/}
		child { node {users}
			child { node {isokhona}
				child { node {formalisms}}
				child { node  {models}
				   child { node [selected]{philosophers}}
				}
				child [missing] {}
				child { node {services}}
				child { node {executions}}
			}
			child [missing] {}
			child [missing] {}
			child [missing] {}
			child [missing] {}
			child [missing] {}
			child { node {alban}}
		}
		child [missing] {}
		child [missing] {}
		child [missing] {}
		child [missing] {}
		child [missing] {}
		child [missing] {}
		child [missing] {}
		child { node {projects}};
}%

\only<23>{%
	\node {/}
		child { node {users}
			child { node {isokhona}
				child { node {formalisms}}
				child { node  {models}}
				child { node {services}}
				child { node {executions}}
				child { node [selected]{projects}}
				child { node [selected]{invitations}}
			}
			child [missing] {}
			child [missing] {}
			child [missing] {}
			child [missing] {}
			child [missing] {}
			child [missing] {}
			child { node {alban}}
		}
		child [missing] {}
		child [missing] {}
		child [missing] {}
		child [missing] {}
		child [missing] {}
		child [missing] {}
		child [missing] {}
		child [missing] {}
		child { node {projects}};
}%

\only<24>{%
	\node {/}
		child { node {users}
			child { node {isokhona}
				child { node {projects}
					child { node [selected]{vfsr\_groupe1}}
				}
			}
		}
		child [missing] {}
		child [missing] {}
		child [missing] {}
		child { node {projects}
			child { node [selected]{vfsr\_groupe1}
				child { node {formalisms}}
				child { node  {models}}
				child { node {users}
					child { node [selected]{isokhona}}
				}
			}
		};
}%

\only<25>{%
	\node {/}
		child { node {users}
			child { node {isokhona}
				child { node {models}
				    child { node [selected]{philosophers}}
				}
			}
		}
		child [missing] {}
		child [missing] {}
		child [missing] {}
		child { node {projects}
			child { node [selected]{vfsr\_groupe1}
				child { node {formalisms}}
				child { node  {models}}
			}
		};
}%

\only<26>{%
	\node {/}
		child { node {users}
			child { node {isokhona}
				child { node {models}}
			}
			child [missing] {}
			child { node {alban}}
		}
		child [missing] {}
		child [missing] {}
		child [missing] {}
		child { node {projects}
			child { node {vfsr\_groupe1}
				child { node {formalisms}}
				child { node  {models}
				      child { node [selected]{philosophers}}
				}
			}
		};
}%

\only<27>{%
	\node {/}
		child { node {users}
		          child { node {isokhona}}
			child { node [selected]{alban}
				child { node {projects}}
				child { node {invitations}}
			}
		}
		child [missing] {}
		child [missing] {}
		child [missing] {}
		child [missing] {}
		child { node {projects}
			child { node [selected] {vfsr\_groupe1}
			          child { node  {users}
				      child { node {isokhona}}
				}
			          child [missing] {}
			          child { node {invitations}}
			}
		};
}%

\only<28>{%
	\node {/}
		child { node {users}
		          child { node {isokhona}}
			child { node {alban}
				child { node {invitations}
					child { node [selected]{vfsr\_groupe1}}
				}
			}
		}
		child [missing] {}
		child [missing] {}
		child [missing] {}
		child [missing] {}
		child { node {projects}
			child { node {vfsr\_groupe1}
			          child { node {invitations}
			          		child { node [selected]{alban}}
			          }
			}
		};
}%

\only<29>{%
	\node {/}
		child { node {users}
		          child { node {isokhona}}
			child { node {alban}
				child { node {projects}
					child { node [selected]{vfsr\_groupe1}}
				}
			}
		}
		child [missing] {}
		child [missing] {}
		child [missing] {}
		child [missing] {}
		child { node {projects}
			child { node [selected] {vfsr\_groupe1}
			          child { node  {users}
				      child { node {isokhona}}
				      child { node [selected]{alban}}
				}
			}
		};
}%


\end{tikzpicture}


\end{column}%
\end{columns}
%\end{minipage}
\note{
  \begin{itemize}
    \item Je vous présente ici au moyen d'exemples les fonctionnalités que
      j'ai implémenté dans le serveur
    \item Explication: à droite l'arborescence des ressources disponibles
      sur le server, à gauche les requêtes et réponses
    \item Ces requêtes sont des requêtes HTTP, elles peuvent donc être
      réalisées depuis n'importe quelle application disposant d'une
      bibliothèque HTTP
    \item Cas 1: création d'un utilisateur
      \begin{itemize}
        \item Sans les droits
        \item Avec les droits
        \item Un utilisateur particulier: l'administrateur
      \end{itemize}
    \item Cas 2: liste des utilisateurs
      \begin{itemize}
        \item Deux types de ressources: les listes et les ressources simples
        \item Distinguées facilement
        \item Fonctionnement similaire pour lister n'importe quelle
          collection de ressources
      \end{itemize}
    \item Cas 3: ajout d'un modèle
      \begin{itemize}
        \item Chaque utilisateur dispose de ressources: formalismes,
          modèles, services et exécutions
        \item Lui seul peut créer ou modifier une ressource
      \end{itemize}
    \item Cas 3-1: ajout d'un autre modèle, puis suppression
    \item Cas 3-2: .... ?
    \item Cas 4: ajout d'un autre utilisateur
    \item Cas 4-1:
      \begin{itemize}
        \item Passage en compte privé d'un utilisateur
        \item Tentative d'accéder à un modèle par cet utilisateur,
          par l'autre utilisateur (échec)
      \end{itemize}
    \item Cas 5: les deux utilisateurs veulent travailler ensemble sur un
      projet industriel
      \begin{itemize}
        \item Impossible car les modèles doivent être privés mais partagés
          entre les deux utilisateurs
        \item Notion de projet (voir slide suivant)
      \end{itemize}
  \end{itemize}
}
\end{frame}

%     \begin{frame}[c]
%  \frametitle{Arborescence Projets}
%  
%  \begin{minipage}{1\textwidth}
%  	\only <1>{%
%                \begin{tikzpicture}
%                      \node
%                          [anchor=north]
%                          at (0,0)
%                          {\includegraphics[width=10.5cm, height=4cm]{img/1-arbre-u}};
%                 \end{tikzpicture}
%            }%
%            
%            \only <2>{%
%                \begin{tikzpicture}
%                      \node
%                          [anchor=north]
%                          at (0,0)
%                          {\includegraphics[width=10.5cm, height=4cm]{img/1-arbre-p}};
%                 \end{tikzpicture}
%            }%
%            
%            \only <3->{%
%                \begin{tikzpicture}
%                      \node
%                          [anchor=north]
%                          at (0,0)
%                          {\includegraphics[width=10.5cm, height=4cm]{img/2-arbre-p}};
%                 \end{tikzpicture}
%            }%
%            
% 
%    \end{minipage}
%   
%   \hrule
%  
%  \begin{minipage}[t][15cm][t]{\textheight}
%  
%  \begin{columns}[T] % align columns
%\begin{column}{.48\textwidth}
%\only<2->{%
%\lstinline!GET http://rest.cosyverif.org/users! \\
%\lstinline!200 OK! \\
%\lstinline!{ ... }!
%}%
%
%\end{column}%
%\hfill%
%\begin{column}{.48\textwidth}
%% http://www.texample.net/tikz/examples/filesystem-tree/
%\tikzstyle{every node}=[draw=black,thick,anchor=west,minimum width=2cm,minimum height=2em]
%\tikzstyle{selected}=[draw=red,fill=red!30]
%\tikzstyle{optional}=[dashed,fill=gray!50]
%\scriptsize
%\begin{tikzpicture}[
%every node={draw=black,thick,anchor=west},
%selected={draw=red,fill=red!30},
%grow via three points={one child at (-.5,-0.7) and
%two children at (-.5,-0.7) and (-.5,-1.4)},
%edge from parent path={(\tikzparentnode.south west) |- (\tikzchildnode.west)}]
%\only<2->{
%\node {/}
%child { node {projects} }
%child { node {users} };
%}
%\end{tikzpicture}
%\end{column}%
%\end{columns}
%    
%   \end{minipage}
% \end{frame}
 
 
\begin{frame}[c]
\only<1>{
  \frametitle{Technologies}
}\only<2>{
	\frametitle{Couverture de test}
}

\only<1>{
\centering
\begin{tikzpicture}
  \node at (-3,2) {\includegraphics[height=1.5cm]{img/php.png}};
  \node at (-1,2) {\includegraphics[height=1.5cm]{img/slim}};
  \node at ( 1,2) {\includegraphics[height=1.5cm]{img/composer}};
  \node at ( 3,2) {\includegraphics[height=1.5cm]{img/box}};
  \node at ( 0,0) {\includegraphics[height=.75cm]{img/docco}};
  \node at (-3,-2) {\includegraphics[height=1.5cm]{img/phpunit}};
  \node at (-.5,-2) {\includegraphics[height=1.5cm]{img/guzzle}};
  \node at ( 3,-2) {\includegraphics[height=1.5cm]{img/travis}};
 \end{tikzpicture} 
}

\only<2>{
  \begin{center}
  	\textsf{80 \%}
   \end{center}
}

%  \begin{minipage}{1\textwidth}
%  	\only <1>{%
%                \begin{tikzpicture}
%                      \node
%                          [anchor=north]
%                          at (0,0)
%                          {\includegraphics[width=10.5cm, height=4cm]{img/1-implementation}};
%                 \end{tikzpicture}
%            }%
%            
%            \only <2->{%
%                \begin{tikzpicture}
%                      \node
%                          [anchor=north]
%                          at (0,0)
%                          {\includegraphics[width=10.5cm, height=4cm]{img/2-implementation}};
%                 \end{tikzpicture}
%            }%
%
%             
%    \end{minipage}
%   
%   \hrule
%  
%  \begin{minipage}{1\textwidth}
%    
%         \begin{itemize}
%     	    \item <1-> Technologies
%     	    \item <2-> Test
%         \end{itemize}
%   \end{minipage}
\note{
  \begin{itemize}
    \item Pour la première ligne : technologies utilisées par le serveur
      \begin{itemize}
      \item PHP: langage de script utilisé pour le code du serveur
      \item Slim: framework permettant d'écrire des applications REST en PHP
      \item Composer: gestionnaire de dépendances pour PHP
      \item Box: Outil de packaging d'une application PHP + dépendances +
        ressources
    \end{itemize}
    \item Seconde ligne : technologie utilisée pour la documentation
      \begin{itemize}
        \item Docco: écriture de la documentation dans le code (literate
          programming)
        \item Génération de documentation HTML
      \end{itemize}
    \item Troisième ligne: technologies utilisées pour le test
      \begin{itemize}
        \item PHPUnit: outil de lancement des tests
        \item On réalise les tests en boîte noire, juste en envoyant des
          requêtes
        \item Guzzle: client Web en PHP permettant de lancer les tests en
          boîte noire
        \item Travis: service en ligne d'intégration continue,
          qui exécute les tests
      \end{itemize}
    \item Couverture de code par les tests: bonne, mais encore quelques
      points à améliorer
  \end{itemize}
}
\end{frame}

%\begin{frame}[c]
%
%\begin{center}
%
%\par
%\Huge Perspectives \& Conclusion
%
%\end{center}
%  
%\end{frame}

\begin{frame}[c]
\only<1>{
  \frametitle{État actuel}
}\only<2>{
  \frametitle{Développement imprévu...}
}
\centering
\begin{tikzpicture}
  \useasboundingbox [] (.5,1) rectangle (10,8.5);
\node (c1) at (2,3)
  {\includegraphics[width=3cm,height=3cm]{img/screen}};
\node (c2) at (2,7)
  {\includegraphics[width=3cm,height=3cm]{img/screen}};
\node at (2,5)
  {Clients};
\node (s) at (7,5)
  {\includegraphics[width=3cm,height=3cm]{img/screen}};
\node [anchor=center]
  at (s.south)
  {\includegraphics[width=1cm]{img/php.png}};
\node [anchor=north west] at (s.east)
  {\includegraphics[width=1cm]{img/depot2}};
\node [anchor=north west] at (s.north east)
  {\includegraphics[width=1cm]{img/users}};
\node at (7,3)
  {Server};
\only<1>{
\node (coloane)
  at (2, 3.3)
  {\color{white}?};
}\only<2->{
\node (webclient)
  at (2, 3.3)
  {\includegraphics[width=2.8cm,height=2.2cm]{img/webclient}};
\node [anchor=center]
  at (c1.south)
  {\includegraphics[width=2cm]{img/html5}};
}
\node
  at (2, 7.3)
  {\includegraphics[width=2.8cm,height=2.2cm]{img/clitor}};
\node (t1)
  at (6.5, 5)
  {\includegraphics[height=1cm]{img/tools-1}};
\node (t2)
  at (7.5, 5.5)
  {\includegraphics[height=1cm]{img/tools-2}};
\draw [<->, line width=2pt]
  (c1) -- (s)
  node [midway, below, sloped] {REST};
\draw [<->, line width=2pt]
  (c2) -- (s)
  node [midway, above, sloped] {REST};
\end{tikzpicture}

%  \begin{minipage}{1\textwidth}
%  	\only <1>{%
%                \begin{tikzpicture}
%                      \node
%                          [anchor=north]
%                          at (0,0)
%                          {\includegraphics[width=10.5cm, height=4cm]{img/1-probleme}};
%                 \end{tikzpicture}
%            }%
%            
%            \only <2->{%
%                \begin{tikzpicture}
%                      \node
%                          [anchor=north]
%                          at (0,0)
%                          {\includegraphics[width=10.5cm, height=4cm]{img/1-client}};
%                 \end{tikzpicture}
%            }%
%
%             
%    \end{minipage}
%   
%   \hrule
%  
%  \begin{minipage}{1\textwidth}
%    
%         \begin{itemize}
%     	    \item <1-> Maintenance
%     	    \item <1-> Fonctionnalités
%	    \item <1-> Evolution
%	    \item <2-> Nouvelle interface web
%         \end{itemize}
%   \end{minipage}
 \end{frame}
 
 
\begin{frame}[c]

\only<1>{%
\frametitle{Page d'accueil}
}%

\only<2-5>{%
\frametitle{Create user}
}%

\only<6>{%
\frametitle{User}
}%

\only<7-11>{%
\frametitle{Create a model}
}%

\only<12-14>{%
\frametitle{Delete a model}
}%

\only<15-18>{%
\frametitle{Private Account}
}%

\only<19-21>{%
\frametitle{Create project}
}%

\only<22>{%
\frametitle{Project}
}%

\only<23-27>{%
\frametitle{Invitations}
}%

\only<28-31>{%
\frametitle{Change permissions}
}%

\only<32->{%
\frametitle{Search}
}%


\only <1>{%
    \begin{tikzpicture}
          \node
              [anchor=north]
              at (0,0)
              {\includegraphics[width=10cm, height=8cm]{img/1-ecran-accueil}};
     \end{tikzpicture}
}%

\only <2>{%
    \begin{tikzpicture}
          \node
              [anchor=north]
              at (0,0)
              {\includegraphics[width=10cm, height=8cm]{img/1-ecran-accueil}};
     \end{tikzpicture}
}%

\only <3>{%
    \begin{tikzpicture}
          \node
              [anchor=north]
              at (0,0)
              {\includegraphics[width=10cm, height=8cm]{img/2-ecran-create-user}};
     \end{tikzpicture}
}%

\only <4>{%
    \begin{tikzpicture}
          \node
              [anchor=north]
              at (0,0)
              {\includegraphics[width=10cm, height=8cm]{img/3-ecran-create-user}};
     \end{tikzpicture}
}%

\only <5>{%
    \begin{tikzpicture}
          \node
              [anchor=north]
              at (0,0)
              {\includegraphics[width=10cm, height=8cm]{img/2-ecran-accueil}};
     \end{tikzpicture}
}%

\only <6>{%
    \begin{tikzpicture}
          \node
              [anchor=north]
              at (0,0)
              {\includegraphics[width=10cm, height=8cm]{img/6-ecran-create-user}};
     \end{tikzpicture}
}%

\only <7>{%
    \begin{tikzpicture}
          \node
              [anchor=north]
              at (0,0)
              {\includegraphics[width=10cm, height=8cm]{img/6-ecran-create-user}};
     \end{tikzpicture}
}%

\only <8>{%
    \begin{tikzpicture}
          \node
              [anchor=north]
              at (0,0)
              {\includegraphics[width=10cm, height=8cm]{img/1-ecran-create-resource}};
     \end{tikzpicture}
}%

\only <9>{%
    \begin{tikzpicture}
          \node
              [anchor=north]
              at (0,0)
              {\includegraphics[width=10cm, height=8cm]{img/2-ecran-create-resource}};
     \end{tikzpicture}
}%

\only <10>{%
    \begin{tikzpicture}
          \node
              [anchor=north]
              at (0,0)
              {\includegraphics[width=10cm, height=8cm]{img/3-ecran-create-resource}};
     \end{tikzpicture}
}%

\only <11>{%
    \begin{tikzpicture}
          \node
              [anchor=north]
              at (0,0)
              {\includegraphics[width=10cm, height=8cm]{img/4-ecran-create-resource}};
     \end{tikzpicture}
}%

\only <12>{%
    \begin{tikzpicture}
          \node
              [anchor=north]
              at (0,0)
              {\includegraphics[width=10cm, height=8cm]{img/1-ecran-delete-model}};
     \end{tikzpicture}
}%

\only <13>{%
    \begin{tikzpicture}
          \node
              [anchor=north]
              at (0,0)
              {\includegraphics[width=10cm, height=8cm]{img/5-ecran-create-resource}};
     \end{tikzpicture}
}%

\only <14>{%
    \begin{tikzpicture}
          \node
              [anchor=north]
              at (0,0)
              {\includegraphics[width=10cm, height=8cm]{img/4-ecran-create-resource}};
     \end{tikzpicture}
}%

\only <15>{%
    \begin{tikzpicture}
          \node
              [anchor=north]
              at (0,0)
              {\includegraphics[width=10cm, height=8cm]{img/1-ecran-public}};
     \end{tikzpicture}
}%

\only <16>{%
    \begin{tikzpicture}
          \node
              [anchor=north]
              at (0,0)
              {\includegraphics[width=10cm, height=8cm]{img/2-ecran-public}};
     \end{tikzpicture}
}%

\only <17>{%
    \begin{tikzpicture}
          \node
              [anchor=north]
              at (0,0)
              {\includegraphics[width=10cm, height=8cm]{img/3-ecran-public}};
     \end{tikzpicture}
}%

\only <18>{%
    \begin{tikzpicture}
          \node
              [anchor=north]
              at (0,0)
              {\includegraphics[width=10cm, height=8cm]{img/4-ecran-public}};
     \end{tikzpicture}
}%

\only <19>{%
    \begin{tikzpicture}
          \node
              [anchor=north]
              at (0,0)
              {\includegraphics[width=10cm, height=8cm]{img/1-ecran-create-project}};
     \end{tikzpicture}
}%

\only <20>{%
    \begin{tikzpicture}
          \node
              [anchor=north]
              at (0,0)
              {\includegraphics[width=10cm, height=8cm]{img/2-ecran-create-project}};
     \end{tikzpicture}
}%

\only <21>{%
    \begin{tikzpicture}
          \node
              [anchor=north]
              at (0,0)
              {\includegraphics[width=10cm, height=8cm]{img/3-ecran-create-project}};
     \end{tikzpicture}
}%

\only <22>{%
    \begin{tikzpicture}
          \node
              [anchor=north]
              at (0,0)
              {\includegraphics[width=10cm, height=8cm]{img/4-ecran-create-project}};
     \end{tikzpicture}
}%

\only <23>{%
    \begin{tikzpicture}
          \node
              [anchor=north]
              at (0,0)
              {\includegraphics[width=10cm, height=8cm]{img/1-ecran-invitation}};
     \end{tikzpicture}
}%

\only <24>{%
    \begin{tikzpicture}
          \node
              [anchor=north]
              at (0,0)
              {\includegraphics[width=10cm, height=8cm]{img/2-ecran-invitation}};
     \end{tikzpicture}
}%

\only <25>{%
    \begin{tikzpicture}
          \node
              [anchor=north]
              at (0,0)
              {\includegraphics[width=10cm, height=8cm]{img/3-ecran-invitation}};
     \end{tikzpicture}
}%

\only <26>{%
    \begin{tikzpicture}
          \node
              [anchor=north]
              at (0,0)
              {\includegraphics[width=10cm, height=8cm]{img/4-ecran-invitation}};
     \end{tikzpicture}
}%

\only <27>{%
    \begin{tikzpicture}
          \node
              [anchor=north]
              at (0,0)
              {\includegraphics[width=10cm, height=8cm]{img/5-ecran-invitation}};
     \end{tikzpicture}
}%

\only <28>{%
    \begin{tikzpicture}
          \node
              [anchor=north]
              at (0,0)
              {\includegraphics[width=10cm, height=8cm]{img/1-ecran-permission}};
     \end{tikzpicture}
}%

\only <29>{%
    \begin{tikzpicture}
          \node
              [anchor=north]
              at (0,0)
              {\includegraphics[width=10cm, height=8cm]{img/2-ecran-permission}};
     \end{tikzpicture}
}%

\only <30>{%
    \begin{tikzpicture}
          \node
              [anchor=north]
              at (0,0)
              {\includegraphics[width=10cm, height=8cm]{img/3-ecran-permission}};
     \end{tikzpicture}
}%

\only <31>{%
    \begin{tikzpicture}
          \node
              [anchor=north]
              at (0,0)
              {\includegraphics[width=10cm, height=8cm]{img/4-ecran-permission}};
     \end{tikzpicture}
}%


\only <32>{%
    \begin{tikzpicture}
          \node
              [anchor=north]
              at (0,0)
              {\includegraphics[width=10cm, height=8cm]{img/1-ecran-search}};
     \end{tikzpicture}
}%

\only <33>{%
    \begin{tikzpicture}
          \node
              [anchor=north]
              at (0,0)
              {\includegraphics[width=10cm, height=8cm]{img/2-ecran-search}};
     \end{tikzpicture}
}%

\only <34>{%
    \begin{tikzpicture}
          \node
              [anchor=north]
              at (0,0)
              {\includegraphics[width=10cm, height=8cm]{img/3-ecran-search}};
     \end{tikzpicture}
}%

\only <35>{%
    \begin{tikzpicture}
          \node
              [anchor=north]
              at (0,0)
              {\includegraphics[width=10cm, height=8cm]{img/4-ecran-search}};
     \end{tikzpicture}
}%

\end{frame}

\begin{frame}[c]
\frametitle{Technologies}
\centering
\begin{tikzpicture}
  \node at (0,0) {\includegraphics[height=3cm]{img/html5}};
  \node at (-3,-3) {\includegraphics[height=1.5cm]{img/bootstrap}};
  \node at ( 0,-3) {\includegraphics[height=1.5cm]{img/jquery}};
  \node at ( 3,-3) {\includegraphics[height=1.5cm]{img/handlebars}};
\end{tikzpicture}

\note{
  \begin{itemize}
    \item Client écrit avec des technologies Web standard: HTML, JavaScript,
      CSS
    \item Utilisation du framework Bootstrap qui fournit le placement en
      grille et des composants utiles
    \item Utilisation de la bibliothèque javascript JQuery, très répandue
    \item Utilisation de templates de HTML5 avec handlebars pour une
      meilleure réutilisation du code
  \end{itemize}
}

%  \begin{minipage}{1\textwidth}
%  	\only <1->{%
%                \begin{tikzpicture}
%                      \node
%                          [anchor=north]
%                          at (0,0)
%                          {\includegraphics[width=10.5cm, height=4cm]{img/1-implementation-client}};
%                 \end{tikzpicture}
%            }%
%            
%    \end{minipage}
%   
%   \hrule
%  
%  \begin{minipage}{1\textwidth}
%    
%         \begin{itemize}
%     	    \item <1-> Technologies
%         \end{itemize}
%   \end{minipage}
\end{frame}

\begin{frame}[c]
  \frametitle{Conclusion}
  
  \only <1>{
  	\begin{center}
		\Huge Bilan
  	\end{center}
  }\only <2>{
  	\begin{center}
		\Huge Perspectives
  	\end{center}
	\begin{block}{Court terme}
    \begin{itemize}
      \item Améliorer la couverture de code
      \item Intégrer l'éditeur collaboratif de modèles
      \item Réaliser un script d'installation
    \end{itemize}
  \end{block}

  \vfill

  \begin{block}{Moyen terme}
    \begin{itemize}
      \item Ajouter la notion de \emph{classe d'étudiants}
      \item Développer des clients mobiles \& tablette
    \end{itemize}
  \end{block}

  }\only <3>{
  	\begin{center}
		\Huge Questions ?
  	\end{center}
  }
  
  \note{
  \begin{itemize}
    \item Ce que tu as appris comme technologies
    \item Ce que tu as appris en méthode de développement
    \item Ce que tu as appris sur la gestion d'un projet
    \item Si c'était à refaire, que changerais-tu ?
    \item Quel avenir professionnel envisages-tu ?
  \end{itemize}
}
\end{frame}

\end{document}
